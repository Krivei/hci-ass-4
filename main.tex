%%
%% This is file `sample-acmsmall-tagged.tex',
%% generated with the docstrip utility.
%%
%% The original source files were:
%%
%% samples.dtx  (with options: `all,journal,acmsmall,tagged')
%% 
%% IMPORTANT NOTICE:
%% 
%% For the copyright see the source file.
%% 
%% Any modified versions of this file must be renamed
%% with new filenames distinct from sample-acmsmall-tagged.tex.
%% 
%% For distribution of the original source see the terms
%% for copying and modification in the file samples.dtx.
%% 
%% This generated file may be distributed as long as the
%% original source files, as listed above, are part of the
%% same distribution. (The sources need not necessarily be
%% in the same archive or directory.)
%%
%%
%% Commands for TeXCount
%TC:macro \cite [option:text,text]
%TC:macro \citep [option:text,text]
%TC:macro \citet [option:text,text]
%TC:envir table 0 1
%TC:envir table* 0 1
%TC:envir tabular [ignore] word
%TC:envir displaymath 0 word
%TC:envir math 0 word
%TC:envir comment 0 0
%%
%% The tagged file should start with the metadata commands.
%% We also need currently use lualatex-dev for compilation!
%% The first command in your LaTeX source must be the \documentclass
%% command.
%%
%% For submission and review of your manuscript please change the
%% command to \documentclass[manuscript, screen, review]{acmart}.
%%
%% When submitting camera ready or to TAPS, please change the command
%% to \documentclass[sigconf]{acmart} or whichever template is required
%% for your publication.
%%
%%
\documentclass[sigchi]{acmart}
%%
%% \BibTeX command to typeset BibTeX logo in the docs
\AtBeginDocument{%
  \providecommand\BibTeX{{%
    Bib\TeX}}}

%% Rights management information.  This information is sent to you
%% when you complete the rights form.  These commands have SAMPLE
%% values in them; it is your responsibility as an author to replace
%% the commands and values with those provided to you when you
%% complete the rights form.
\setcopyright{acmlicensed}
\copyrightyear{2018}
\acmYear{2018}
\acmDOI{XXXXXXX.XXXXXXX}

%%
%% These commands are for a JOURNAL article.
\acmJournal{JACM}
\acmVolume{37}
\acmNumber{4}
\acmArticle{111}
\acmMonth{8}

%%
%% Submission ID.
%% Use this when submitting an article to a sponsored event. You'll
%% receive a unique submission ID from the organizers
%% of the event, and this ID should be used as the parameter to this command.
%%\acmSubmissionID{123-A56-BU3}

%%
%% For managing citations, it is recommended to use bibliography
%% files in BibTeX format.
%%
%% You can then either use BibTeX with the ACM-Reference-Format style,
%% or BibLaTeX with the acmnumeric or acmauthoryear sytles, that include
%% support for advanced citation of software artefact from the
%% biblatex-software package, also separately available on CTAN.
%%
%% Look at the sample-*-biblatex.tex files for templates showcasing
%% the biblatex styles.
%%

%% use of package for code snippets
\usepackage{tabularx}
\usepackage{subcaption}
\usepackage{float}
\usepackage{multirow}
\usepackage{array}
%%
%% end of the preamble, start of the body of the document source.
\begin{document}

%%
%% The "title" command has an optional parameter,
%% allowing the author to define a "short title" to be used in page headers.
\title{Managing Attention Across Household Appliances During Cooking}

%%
%% The "author" command and its associated commands are used to define
%% the authors and their affiliations.
%% Of note is the shared affiliation of the first two authors, and the
%% "authornote" and "authornotemark" commands
%% used to denote shared contribution to the research.
\author{Lah Hong Wai}
\affiliation{%
	\institution{Bauhaus-Universität Weimar}
	\city{Weimar}
	\country{Germany}}
\email{lah.hong.wai@uni-weimar.de}

\author{Daniel Radianto}
\affiliation{%
	\institution{Bauhaus-Universität Weimar}
	\city{Weimar}
	\country{Germany}}
\email{daniel.cristianindra.radianto@uni-weimar.de}

\author{Xavier Theodosius}
\affiliation{%
	\institution{Bauhaus-Universität Weimar}
	\city{Weimar}
	\country{Germany}}
\email{xavier.julian.theodosius@uni-weimar.de}


%%
%% By default, the full list of authors will be used in the page
%% headers. Often, this list is too long, and will overlap
%% other information printed in the page headers. This command allows
%% the author to define a more concise list
%% of authors' names for this purpose.
\renewcommand{\shortauthors}{Lah et al.}

%%
%% The abstract is a short summary of the work to be presented in the
%% article.
\begin{abstract}
	Cooking is a complex activity that requires people to coordinate multiple appliances while managing time, safety, and task demands in their personal space. Understanding how attention is distributed during cooking is important for Human-Computer Interaction, as breakdowns in attention can lead to errors or unsafe situations. Using remote re-enactment video as a method, this study investigates how individuals distribute attention and interaction across kitchen appliances during cooking. Despite limitations such as a restricted field of view and limited contextual information, the collected data allowed observation of cooking sequences, attention shifts, and task transitions. Qualitative thematic analysis revealed recurring patterns in how participants coordinated actions, managed timing, and maintained situational awareness while cooking. The findings highlight that attention management relies on continuous adjustment rather than constant focus, and indicate that both timing and situational awareness play important roles in shaping the cooking workflow, as multiple tasks are carried out in parallel and often depend on one another. Collectively, cooking heavily emphasizes maintaining a workflow through continuous shifts in attention and interaction across appliances.
\end{abstract}

%%
%% The code below is generated by the tool at http://dl.acm.org/ccs.cfm.
%% Please copy and paste the code instead of the example below.
%%
\begin{CCSXML}
	<ccs2012>
	<concept>
	<concept_id>10003120.10003121.10011748</concept_id>
	<concept_desc>Human-centered computing~Empirical studies in HCI</concept_desc>
	<concept_significance>500</concept_significance>
	</concept>
	<concept>
	<concept_id>10003120.10003121.10003122.10003334</concept_id>
	<concept_desc>Human-centered computing~User studies</concept_desc>
	<concept_significance>500</concept_significance>
	</concept>
	<concept>
	<concept_id>10003120.10003123</concept_id>
	<concept_desc>Human-centered computing~Interaction design</concept_desc>
	<concept_significance>300</concept_significance>
	</concept>
	<concept>
	<concept_id>10003120.10003121.10003126</concept_id>
	<concept_desc>Human-centered computing~HCI theory, concepts and models</concept_desc>
	<concept_significance>300</concept_significance>
	</concept>
	<concept>
	<concept_id>10003120.10003121.10003124.10011751</concept_id>
	<concept_desc>Human-centered computing~Collaborative interaction</concept_desc>
	<concept_significance>300</concept_significance>
	</concept>
	</ccs2012>
\end{CCSXML}

\ccsdesc[500]{Human-centered computing~Empirical studies in HCI}
\ccsdesc[500]{Human-centered computing~User studies}
\ccsdesc[300]{Human-centered computing~Interaction design}
\ccsdesc[300]{Human-centered computing~HCI theory, concepts and models}
\ccsdesc[300]{Human-centered computing~Collaborative interaction}

%%
%% Keywords. The author(s) should pick words that accurately describe
%% the work being presented. Separate the keywords with commas.
\keywords{Waiting in Line, Behavioural Tendency, Behaviour Pattern, Qualitative Analysis}

%%
%% This command processes the author and affiliation and title
%% information and builds the first part of the formatted document.
\maketitle

\section{Introduction}

    Cooking is an integral activity in daily life. A combination of appliances can be used for this activity, including sanitization, food preparation, and serving. Using kitchen appliances during cooking requires a certain level of attention. Although cooking does not require constant focus, interruptions or shifts in attention can lead to mistakes or safety issues if the ongoing cooking process is forgotten. Maintaining awareness of active appliances, particularly those that involve heat or sharp objects, is therefore important during cooking. Without proper attention while cooking, accidents such as burnt food, overheated utensils, and even accidental harm can occur due to minor distractions or inattention. Understanding how attention and interaction work during cooking is crucial for Human-Computer Interaction, as cooking is a complex, real-world activity that combines physical actions, cognitive decision-making, and time management. By examining cooking as an everyday practice, this research aims to understand how people naturally manage multiple technologies in their home environment.
  
  This study focuses on how each individual distributes their attention and interacts with appliances. Attention is understood as the mental focus at a given time, while interaction is more about how an individual can create engagement across appliances, such as cleaning, preparing a serving place, or responding to alerts. Rather than treating appliance use as isolated actions, this research views cooking as a dynamic process in which users continuously adjust their focus in response to task demands, the appliance's response, and the current situation. Investigating these patterns can reveal how people prioritize tasks, manage interruptions, and coordinate actions across devices.   
  
  Building on this perspective, the research question guiding this study is: “How do individuals distribute attention and interaction across household appliances during everyday cooking?” This is an exploratory qualitative study that does not aim to establish causal relationships, but to identify recurring patterns of attention distribution and interaction. Using re-enactment video as the chosen qualitative method, the study seeks to understand how attention shifts across appliances and what factors influence these shifts, such as task importance, timing, and perceived risks (e.g., burning food or overcooking). Interactions are expected to concentrate around specific moments, including preparation, monitoring progress, and responding to changes during the cooking process.

\section{Methodology}

Studying how individuals distribute attention across different household appliances during everyday cooking presents challenges, as attention shifts and coordination are difficult to capture in real time. Prior research addressed these challenges by employing re-enacted scenarios to capture interaction and routines \cite{Pink2017Design}. Drawing on insights from previous studies \cite{Buchenau2000}, the present study adopts re-enactment videos (REV), where they capture individuals re-enact a particular scenario from their everyday cooking. Video re-enactment was selected as it was an appropriate method to explore and understand participants' experiences, visible and hidden \cite{Pink2015DoingSensoryEthnography}. Themes were then developed through thematic analysis from the data.

To undertake this research, we focused on understanding how interaction was done with household appliances during cooking, and how attention could be distributed. Particular attention was paid to transitions between commonly used kitchen appliances, such as the refrigerator, stove and sink, which are commonly described as a ``kitchen work triangle''. This concept was only used as a guide to prioritise observation of kitchen appliances, and was not involved in the analytical framework. To adopt this method, we used visual techniques to record the actions, followed by a post-interview review of the recorded videos. During the re-enactment, participants were encouraged to think-aloud, to better understand their actions.

Prior to the main study, a pilot session was conducted to understand the re-enactment flow, video setup and prompting strategy. This session led to minor refinements to the study, including reducing prompting to avoid interruption and finding good angles for recording. Data from the pilot session were not included in the final analysis.

\subsection{Participants}

Recruited participants were initially screened to ensure they had prior experience with everyday home cooking and familiarity with using common household appliances. Three participants took part in the study, ranging in age from 24 to 30 years. Participants were recruited via convenience sampling, based on availability and willingness to take part in the study.

All participants reported cooking at home several times per week. They were all familiar with commonly used kitchen appliances, including a stove, refrigerator, and sink. One participant mentioned cooking for oneself, particularly for meal prep. Another participant reported a shift from cooking for household to only one due to a change in living situation. Cooking for household as a responsibility was noted for the third participant.

This study involved a small number of participants (N = 3) and was conducted in a domestic setting. This was done to focus on gaining in-depth insights in the experiences of individuals, on how attention and interaction are distributed across household appliances during everyday cooking, rather than statistical results.

\subsection{Materials}

Due to the nature of this study, participants were only required to perform re-enactments in their domestic kitchen environment. All participants provided consent prior to the participation. Only a recording device was required by the participant to capture interaction. No specialised equipment was provided to participants for the study. Data were collected remotely using Google Meet, which was used to record video and audio of cooking activities and participant commentary.

Recordings were transcribed using Condens.io to support qualitative analysis. Data were stored securely in a device accessible only to the researchers. To illustrate and pick key interaction sequences to supplement the analysis, video recordings were translated into participant-specific storyboards.

\subsection{Procedure}

During the screening session, participants were requested to select a recipe that they were comfortable with. In the re-enactment, they were asked to perform a cooking scenario based on the recipe. They were allowed to select a time context (e.g. lunch, dinner) which they find suitable for their selected recipe. Little guidance was given, besides giving them examples of the starting and ending point of a cooking process. We allowed these points to be freely selected by the participants, such that we could capture any insightful scenarios that may not be part of a cooking process such as cleaning up after dinner. These scenarios could help explain participant-specific interaction patterns observed during cooking.

This re-enactment was facilitated by one researcher, where limited prompts were given during the session for further understanding. The prompts were asked in each ``break'' \textemdash~ short sessions where participants transitioned from one sub-action to another, such as chopping to cooking \textemdash~ to ensure the process remains uninterrupted. While re-enacting, participants were encouraged to narrate on their interactions. They were also told that they should not reveal, say or answer anything that were felt uncomfortable. After this session, the researchers re-watched the video, noting down important actions and interactions that require more explanation. Another session, but not recorded, was done with the participants to assist in understanding the peculiar parts, supported by the recorded video.

\subsection{Spatial Context of Cooking Activities}

To complement the discussion of spatial movement and appliance coordination during the cooking activities, one participant consented to showing the kitchen floor plan. Therefore, a simplified schematic floor plan of the kitchen was drawn, shown in Figure~\ref{fig:context_1}. The floor plan illustrated only the main appliances that the participant often interact with. Exact dimensions of the kitchen layout were not recorded, as we only wanted to understand relative spatial relationships between appliances rather than precise measurements.

\begin{figure}[H]
	\centering
	\includegraphics[width=\columnwidth]{figures/p1_plan.png}
	\caption{\textbf{Schematic floor plan of the kitchen.} These dimensions were not representative of the owner's actual dimension.}
	\label{fig:context_1}
\end{figure}

Further understanding of the interaction with the stove was needed to support the interpretation of participants' interaction. This presented as a useful aid to actions that the video could not capture, due to viewing area limitations. The same participant also consented to the inclusion of a schematic sketch of the stove interface, shown in Figure~\ref{fig:context_2}.

\begin{figure}[H]
	\centering
	\includegraphics[width=\columnwidth]{figures/p1_cook.png}
	\caption{Schematic of the stove.}
	\label{fig:context_2}
\end{figure}

The stove contained two induction hobs, operated through electricity. The hob is turned on by holding the power button (placed at the right of +) for 2 seconds. The temperature is adjusted with the increase + and decrease - buttons around the display, located between the two buttons. The most left button was a child lock button, to prevent children from accidentally switching on the stove.

These two figures will be used to complement findings that require more elaboration and challenges that the participant had faced.

\subsection{Data Analysis}

Recordings and transcripts were stored in a cloud folder with password protection, and only shared among researchers involved in this study. These materials were used for thematic analysis \cite{BraunClarke2021}, where we identify recurring patterns to understand underlying meanings of participants in their cooking re-enactment. The transcripts were analysed for interesting quotes, retrieved and then placed in Figma for collaboration. 

We went through a five-stage process, with iteration when appropriate, following Braun and Clarke's guidance \cite{BraunClarke2021}. Each interesting quote was designated a code in the first cycle. The second stage involved refining codes, discarding duplicates and removing superfluous quotes. Patterns were found among codes, grouped and themes were later identified.

In addition, the videos were also analysed. Actions involving hands, routes, activities, materials used, senses and cues were recorded. Facial expression was also scrutinised when participants encountered challenges. To facilitate the understanding of relationships between actions and narrations, transcripts were viewed alongside with videos, and actions noted down. Separating these two presents a false construct because the narration may not accurately reflect the actions, frequently done unintentionally or not knowingly, and interesting insights may be omitted. Therefore, both resources were used together for the analysis.

\section{Findings}



%%
%% The acknowledgments section is defined using the "acks" environment
%% (and NOT an unnumbered section). This ensures the proper
%% identification of the section in the article metadata, and the
%% consistent spelling of the heading.
\begin{acks}
	We thank the participants for providing us their time to participate in this research. Special recognition goes to Professor Eva and Margarita Osipova for their guidance on this research study.
\end{acks}

%%
%% The next two lines define the bibliography style to be used, and
%% the bibliography file.
\bibliographystyle{ACM-Reference-Format}
\bibliography{main-base}

%%
%% If your work has an appendix, this is the place to put it.
% \appendix


\end{document}
\endinput
%%
%% End of file `sigconf-main.tex'.