%%
%% This is file `sample-acmsmall-tagged.tex',
%% generated with the docstrip utility.
%%
%% The original source files were:
%%
%% samples.dtx  (with options: `all,journal,acmsmall,tagged')
%% 
%% IMPORTANT NOTICE:
%% 
%% For the copyright see the source file.
%% 
%% Any modified versions of this file must be renamed
%% with new filenames distinct from sample-acmsmall-tagged.tex.
%% 
%% For distribution of the original source see the terms
%% for copying and modification in the file samples.dtx.
%% 
%% This generated file may be distributed as long as the
%% original source files, as listed above, are part of the
%% same distribution. (The sources need not necessarily be
%% in the same archive or directory.)
%%
%%
%% Commands for TeXCount
%TC:macro \cite [option:text,text]
%TC:macro \citep [option:text,text]
%TC:macro \citet [option:text,text]
%TC:envir table 0 1
%TC:envir table* 0 1
%TC:envir tabular [ignore] word
%TC:envir displaymath 0 word
%TC:envir math 0 word
%TC:envir comment 0 0
%%
%% The tagged file should start with the metadata commands.
%% We also need currently use lualatex-dev for compilation!
%% The first command in your LaTeX source must be the \documentclass
%% command.
%%
%% For submission and review of your manuscript please change the
%% command to \documentclass[manuscript, screen, review]{acmart}.
%%
%% When submitting camera ready or to TAPS, please change the command
%% to \documentclass[sigconf]{acmart} or whichever template is required
%% for your publication.
%%
%%
\documentclass[sigchi]{acmart}
%%
%% \BibTeX command to typeset BibTeX logo in the docs
\AtBeginDocument{%
  \providecommand\BibTeX{{%
    Bib\TeX}}}

%% Rights management information.  This information is sent to you
%% when you complete the rights form.  These commands have SAMPLE
%% values in them; it is your responsibility as an author to replace
%% the commands and values with those provided to you when you
%% complete the rights form.
\setcopyright{acmlicensed}
\copyrightyear{2018}
\acmYear{2018}
\acmDOI{XXXXXXX.XXXXXXX}

%%
%% These commands are for a JOURNAL article.
\acmJournal{JACM}
\acmVolume{37}
\acmNumber{4}
\acmArticle{111}
\acmMonth{8}

%%
%% Submission ID.
%% Use this when submitting an article to a sponsored event. You'll
%% receive a unique submission ID from the organizers
%% of the event, and this ID should be used as the parameter to this command.
%%\acmSubmissionID{123-A56-BU3}

%%
%% For managing citations, it is recommended to use bibliography
%% files in BibTeX format.
%%
%% You can then either use BibTeX with the ACM-Reference-Format style,
%% or BibLaTeX with the acmnumeric or acmauthoryear sytles, that include
%% support for advanced citation of software artefact from the
%% biblatex-software package, also separately available on CTAN.
%%
%% Look at the sample-*-biblatex.tex files for templates showcasing
%% the biblatex styles.
%%

%% use of package for code snippets
\usepackage{tabularx}
\usepackage{subcaption}
\usepackage{float}
\usepackage{multirow}
\usepackage{array}
%%
%% end of the preamble, start of the body of the document source.
\begin{document}

%%
%% The "title" command has an optional parameter,
%% allowing the author to define a "short title" to be used in page headers.
\title{Managing Attention Across Household Appliances During Cooking}

%%
%% The "author" command and its associated commands are used to define
%% the authors and their affiliations.
%% Of note is the shared affiliation of the first two authors, and the
%% "authornote" and "authornotemark" commands
%% used to denote shared contribution to the research.
\author{Lah Hong Wai}
\affiliation{%
	\institution{Bauhaus-Universität Weimar}
	\city{Weimar}
	\country{Germany}}
\email{lah.hong.wai@uni-weimar.de}

\author{Daniel Radianto}
\affiliation{%
	\institution{Bauhaus-Universität Weimar}
	\city{Weimar}
	\country{Germany}}
\email{daniel.cristianindra.radianto@uni-weimar.de}

\author{Xavier Theodosius}
\affiliation{%
	\institution{Bauhaus-Universität Weimar}
	\city{Weimar}
	\country{Germany}}
\email{xavier.julian.theodosius@uni-weimar.de}


%%
%% By default, the full list of authors will be used in the page
%% headers. Often, this list is too long, and will overlap
%% other information printed in the page headers. This command allows
%% the author to define a more concise list
%% of authors' names for this purpose.
\renewcommand{\shortauthors}{Lah et al.}

%%
%% The abstract is a short summary of the work to be presented in the
%% article.
\begin{abstract}
	Cooking is an integral activity in daily life. A combination of appliances can be used for this activity, including sanitization, food preparation, and serving. The use of kitchen appliances during cooking requires a certain level of attentional awareness. Although cooking does not require constant focus at all times, interruptions or shifts of attention can lead to mistakes or safety issues if the ongoing cooking process is forgotten. Maintaining awareness of active appliances, particularly those involving heat or sharp object, is therefore important during cooking activities. Without proper attention while cooking, accidents such as burnt food, overheated utensils, and even accidental-harm can occur due to minor distractions or inattention. Understanding how attention and interaction work during cooking is very important for Human-Computer Interaction, as cooking is a complex, real-world activity that combines physical action, cognitive decision-making, and time management. By examining cooking as an everyday practice, this research aims to understand how people naturally manage multiple technologies in their home environment.
\end{abstract}

%%
%% The code below is generated by the tool at http://dl.acm.org/ccs.cfm.
%% Please copy and paste the code instead of the example below.
%%
\begin{CCSXML}
	<ccs2012>
	<concept>
	<concept_id>10003120.10003121.10011748</concept_id>
	<concept_desc>Human-centered computing~Empirical studies in HCI</concept_desc>
	<concept_significance>500</concept_significance>
	</concept>
	<concept>
	<concept_id>10003120.10003121.10003122.10003334</concept_id>
	<concept_desc>Human-centered computing~User studies</concept_desc>
	<concept_significance>500</concept_significance>
	</concept>
	<concept>
	<concept_id>10003120.10003123</concept_id>
	<concept_desc>Human-centered computing~Interaction design</concept_desc>
	<concept_significance>300</concept_significance>
	</concept>
	<concept>
	<concept_id>10003120.10003121.10003126</concept_id>
	<concept_desc>Human-centered computing~HCI theory, concepts and models</concept_desc>
	<concept_significance>300</concept_significance>
	</concept>
	<concept>
	<concept_id>10003120.10003121.10003124.10011751</concept_id>
	<concept_desc>Human-centered computing~Collaborative interaction</concept_desc>
	<concept_significance>300</concept_significance>
	</concept>
	</ccs2012>
\end{CCSXML}

\ccsdesc[500]{Human-centered computing~Empirical studies in HCI}
\ccsdesc[500]{Human-centered computing~User studies}
\ccsdesc[300]{Human-centered computing~Interaction design}
\ccsdesc[300]{Human-centered computing~HCI theory, concepts and models}
\ccsdesc[300]{Human-centered computing~Collaborative interaction}

%%
%% Keywords. The author(s) should pick words that accurately describe
%% the work being presented. Separate the keywords with commas.
\keywords{Waiting in Line, Behavioural Tendency, Behaviour Pattern, Qualitative Analysis}

%%
%% This command processes the author and affiliation and title
%% information and builds the first part of the formatted document.
\maketitle

\section{Introduction}

    Cooking is an integral activity in daily life. A combination of appliances can be used for this activity, including sanitization, food preparation, and serving. Using kitchen appliances during cooking requires a certain level of attention. Although cooking does not require constant focus, interruptions or shifts in attention can lead to mistakes or safety issues if the ongoing cooking process is forgotten. Maintaining awareness of active appliances, particularly those that involve heat or sharp objects, is therefore important during cooking. Without proper attention while cooking, accidents such as burnt food, overheated utensils, and even accidental harm can occur due to minor distractions or inattention. Understanding how attention and interaction work during cooking is crucial for Human-Computer Interaction, as cooking is a complex, real-world activity that combines physical actions, cognitive decision-making, and time management. By examining cooking as an everyday practice, this research aims to understand how people naturally manage multiple technologies in their home environment.
  
  This study focuses on how each individual distributes their attention and interacts with appliances. Attention is understood as the mental focus at a given time, while interaction is more about how an individual can create engagement across appliances, such as cleaning, preparing a serving place, or responding to alerts. Rather than treating appliance use as isolated actions, this research views cooking as a dynamic process in which users continuously adjust their focus in response to task demands, the appliance's response, and the current situation. Investigating these patterns can reveal how people prioritize tasks, manage interruptions, and coordinate actions across devices.   
  
  Building on this perspective, the research question guiding this study is: “How do individuals distribute attention and interaction across household appliances during everyday cooking?” This is an exploratory qualitative study that does not aim to establish causal relationships, but to identify recurring patterns of attention distribution and interaction. Using re-enactment video as the chosen qualitative method, the study seeks to understand how attention shifts across appliances and what factors influence these shifts, such as task importance, timing, and perceived risks (e.g., burning food or overcooking). Interactions are expected to concentrate around specific moments, including preparation, monitoring progress, and responding to changes during the cooking process.

\section{Methodology}

Studying how individuals distribute attention across different household appliances during everyday cooking presents challenges, as attention shifts and coordination are difficult to capture in real time. Prior research addressed these challenges by employing re-enacted scenarios to capture interaction and routines \cite{Pink2017Design}. Drawing on insights from previous studies \cite{Buchenau2000}, the present study adopts re-enactment videos (REV), where they capture individuals re-enact a particular scenario from their everyday cooking. Video re-enactment was selected as it was an appropriate method to explore and understand participants' experiences, visible and hidden \cite{Pink2015DoingSensoryEthnography}. Themes were then developed through thematic analysis from the data.

To undertake this research, we focused on understanding how interaction was done with household appliances during cooking, and how attention could be distributed. Particular attention was paid to transitions between commonly used kitchen appliances, such as the refrigerator, stove and sink, which are commonly described as a ``kitchen work triangle''. This concept was only used as a guide to prioritise observation of kitchen appliances, and was not involved in the analytical framework. To adopt this method, we used visual techniques to record the actions, followed by a post-interview review of the recorded videos. During the re-enactment, participants were encouraged to think-aloud, to better understand their actions.

Prior to the main study, a pilot session was conducted to understand the re-enactment flow, video setup and prompting strategy. This session led to minor refinements to the study, including reducing prompting to avoid interruption and finding good angles for recording. Data from the pilot session were not included in the final analysis.

\subsection{Participants}

Recruited participants were initially screened to ensure they had prior experience with everyday home cooking and familiarity with using common household appliances. Three participants took part in the study, ranging in age from 24 to 30 years. Participants were recruited via convenience sampling, based on availability and willingness to take part in the study.

All participants reported cooking at home several times per week. They were all familiar with commonly used kitchen appliances, including a stove, refrigerator, and sink. One participant mentioned cooking for oneself, particularly for meal prep. Another participant reported a shift from cooking for household to only one due to a change in living situation. Cooking for household as a responsibility was noted for the third participant.

This study involved a small number of participants (N = 3) and was conducted in a domestic setting. This was done to focus on gaining in-depth insights in the experiences of individuals, on how attention and interaction are distributed across household appliances during everyday cooking, rather than statistical results.

\subsection{Materials}

Due to the nature of this study, participants were only required to perform re-enactments in their domestic kitchen environment. All participants provided consent prior to the participation. Only a recording device was required by the participant to capture interaction. No specialised equipment was provided to participants for the study. Data were collected remotely using Google Meet, which was used to record video and audio of cooking activities and participant commentary.

Recordings were transcribed using Condens.io to support qualitative analysis. Data were stored securely in a device accessible only to the researchers. To illustrate and pick key interaction sequences to supplement the analysis, video recordings were translated into participant-specific storyboards.

\subsection{Procedure}

During the screening session, participants were requested to select a recipe that they were comfortable with. In the re-enactment, they were asked to perform a cooking scenario based on the recipe. They were allowed to select a time context (e.g. lunch, dinner) which they find suitable for their selected recipe. Little guidance was given, besides giving them examples of the starting and ending point of a cooking process. We allowed these points to be freely selected by the participants, such that we could capture any insightful scenarios that may not be part of a cooking process such as cleaning up after dinner. These scenarios could help explain participant-specific interaction patterns observed during cooking.

This re-enactment was facilitated by one researcher, where limited prompts were given during the session for further understanding. The prompts were asked in each ``break'' \textemdash~ short sessions where participants transitioned from one sub-action to another, such as chopping to cooking \textemdash~ to ensure the process remains uninterrupted. While re-enacting, participants were encouraged to narrate on their interactions. They were also told that they should not reveal, say or answer anything that were felt uncomfortable. After this session, the researchers re-watched the video, noting down important actions and interactions that require more explanation. Another session, but not recorded, was done with the participants to assist in understanding the peculiar parts, supported by the recorded video.

\subsection{Spatial Context of Cooking Activities}

To complement the discussion of spatial movement and appliance coordination during the cooking activities, one participant consented to showing the kitchen floor plan. Therefore, a simplified schematic floor plan of the kitchen was drawn, shown in Figure~\ref{fig:context_1}. The floor plan illustrated only the main appliances that the participant often interact with. Exact dimensions of the kitchen layout were not recorded, as we only wanted to understand relative spatial relationships between appliances rather than precise measurements.

\begin{figure}[H]
	\centering
	\includegraphics[width=\columnwidth]{figures/p1_plan.png}
	\caption{\textbf{Schematic floor plan of the kitchen.} These dimensions were not representative of the owner's actual dimension.}
	\label{fig:context_1}
\end{figure}

Further understanding of the interaction with the stove was needed to support the interpretation of participants' interaction. This presented as a useful aid to actions that the video could not capture, due to viewing area limitations. The same participant also consented to the inclusion of a schematic sketch of the stove interface, shown in Figure~\ref{fig:context_2}.

\begin{figure}[H]
	\centering
	\includegraphics[width=\columnwidth]{figures/p1_cook.png}
	\caption{Schematic of the stove.}
	\label{fig:context_2}
\end{figure}

The stove contained two induction hobs, operated through electricity. The hob is turned on by holding the power button (placed at the right of +) for 2 seconds. The temperature is adjusted with the increase + and decrease - buttons around the display, located between the two buttons. The most left button was a child lock button, to prevent children from accidentally switching on the stove.

These two figures will be used to complement findings that require more elaboration and challenges that the participant had faced.

\subsection{Data Analysis}

Recordings and transcripts were stored in a cloud folder with password protection, and only shared among researchers involved in this study. These materials were used for thematic analysis \cite{BraunClarke2021}, where we identify recurring patterns to understand underlying meanings of participants in their cooking re-enactment. The transcripts were analysed for interesting quotes, retrieved and then placed in Figma for collaboration. 

We went through a five-stage process, with iteration when appropriate, following Braun and Clarke's guidance \cite{BraunClarke2021}. Each interesting quote was designated a code in the first cycle. The second stage involved refining codes, discarding duplicates and removing superfluous quotes. Patterns were found among codes, grouped and themes were later identified.

In addition, the videos were also analysed. Actions involving hands, routes, activities, materials used, senses and cues were recorded. Facial expression was also scrutinised when participants encountered challenges. To facilitate the understanding of relationships between actions and narrations, transcripts were viewed alongside with videos, and actions noted down. Separating these two presents a false construct because the narration may not accurately reflect the actions, frequently done unintentionally or not knowingly, and interesting insights may be omitted. Therefore, both resources were used together for the analysis.

\section{Findings}
Each theme that were identified after the analysis will be described thoroughly in this section.

\subsection{Coordinating cooking workflows under constraints}
\subsubsection{Sequential and parallel appliance use}
There were actions that must be done sequentially to reach the desired outcome, such as partially cooking minced meat until its 70% cooked, smashing garlic to make the peeling process easier and frying it first before putting meat into the wok. Appliances were also used in parallel to speed up tasks while maintaining performance, such as taking out a spatula to stir and cook, or using it to move vegetables from the wok to a plate.
Observations revealed that individuals would pro-actively organise during the cooking process to maintain workflow. Examples include preparing a bowl while the cooking process is ongoing, storing the extra ingredients in the fridge, and washing a used spatula first to keep it out of the way in another process.

\subsubsection{Unintentionally delaying actions until conditions are met}
During cooking, certain actions were often delayed until specific conditions were met, without the individual explicitly planning to postpone them. For example, ingredients such as lentils were left to soak until they were considered ready, and tools like chopping boards remained unused until a later stage of cooking. These delays might seem insignificant, yet they played an important role in supporting smoother transitions between tasks by reducing the need to repeat preparation steps. In many cases, the delay did not result from conscious decision-making but emerged from dependencies between appliances and limited resources, such as when the sink was still in use and washing dishes or rinsing ingredients had to be postponed. As a result, attention was directed not only to ongoing actions but also to temporarily “on hold” actions, requiring individuals to remain aware of when these postponed tasks could be resumed. This pattern shows how attention during cooking is managed over time, shaped by situational constraints rather than explicit planning.

\subsubsection{Managing unintended delays caused by safety or space constraints}
Another pattern that was identified was the tendency for individuals to unintentionally delay certain processes. The behaviour may happen due to space constraints. For example, individuals were observed pushing bowls around to find space on the counter, finding a position to place a plate down, and rearranging dishes in the dryer to make more space. Another interpretation of such behavior is the individuals concern for their safety. Observation revealed an individual delaying the cutting process to ensure their safety by putting distance between the knife and their hand before cutting an ingredient.

\subsection{Monitoring progress through senses and evaluation}
\subsubsection{Using sensory cues to assess cooking state}
To determine the food quality, individuals tend to rely on their sensory cues during the cooking process. Appliances were managed to prevent deterioration in food quality. Sensory cues include observing texture changes of the ingredients, putting spices when sizzling sound is heard during the cooking process and listening to pressure cooker sound to check for the utensil’s functionality. Sensory cues were not limited to visual or auditory cues. For example, individuals would not only check for smoke to gauge their pan temperature but also use their hand to sense the heat as a tactile cue.

\subsubsection{Reflecting on outcomes to judge success or quality}
In order to obtain their desired outcome, individuals used appliances accurately and precisely. Examples for such a pattern include the instance where an individual reduces the stove heat to a certain level to prevent burning the ingredients, using a lid to cover a wok while setting the stove to its highest temperature, or tossing their vegetables up and down to stir during frying.ring interaction.

\subsubsection{Hygiene focus with appliance, food, and self}
Observations revealed a consistent behaviour of cleaning the appliances, food, and self. The sink was frequently used throughout the cooking session to either clean the appliances, ingredients, or their hands. Not only by washing their hands during the cooking session, an individual would also check for oil residue on their appliance, and would soak it multiple times to make sure that it is clean. Ingredients such as vegetables, or rice, were also cleaned before being processed.

\subsubsection{Appliance selection based on perceived risk}
Another risk-management example is the appliance selection, such as selecting an electric stove instead of a gas stove as they perceived it for being a safer option. In order to compensate for identified problems, individuals were able to find a strategy , such as taking precaution when using a very sharp knife, using two hands to facilitate transferring objects from a heavy wok to the sink, and checking an old pressure cooker for its functionality.

\subsection{Applying knowledge, experience, and strategies to maintain control}
\subsubsection{Switching between experience-based judgement and precise measurement}
During the cooking session, individuals were observed switching between relying on experience-based judgement and the use of precise measurement. As an example, individuals would put spices in their food according to what they perceive as the correct amount. However, there are instances where the individuals relied on precise measurements, such as a table spoon or a food weighing scale.

\subsubsection{Preparing components in advance to reduce error}
In order to maintain a smooth workflow in the cooking session, individuals had the tendency to prepare appliances and components beforehand, such as preparing white rice before the cooking session, defrosting meat for an amount of time before stir-frying them, and preparing the required cooking utensils.

\begin{table}[H]
	\caption{Themes, Patterns, and Example Encodings}
	\label{tab:themetable1}
	\centering
	\renewcommand{\arraystretch}{1.4}
	\begin{tabular}{|p{0.25\linewidth}|p{0.25\linewidth}|p{0.4\linewidth}|}
		\hline
		\textbf{Theme} & \textbf{Pattern Name} & \textbf{Example Encodings} \\ \hline
		
		\multirow{4}{=}{Coordinating cooking workflows under constrains} 
		
		& Sequential actions for desired outcomes 
		& \begin{itemize}
			\item Fry (minced meat) until it's about 70\% cooked
			\item I almost forget. Before frying the veggies, we have to fry the meat first.
			\item smash to ... make ... the outer layers of the garlic to peel more easily
		\end{itemize} \\ \cline{2-3}
		
		& Delaying appliances use to be accomplished later
		& \begin{itemize}
			\item The spatula I will put it into the inside the wok so that I will maybe I'll still need afterwards
			\item Also, Im going to leave my pan and the cutting board I used to cut chicken behind since I have a dishwasher.
			\item Then put some more water in the bowl (filled with lentils) and leave it aside for some time.
		\end{itemize} \\ \cline{2-3}
		
		& Waiting for cookware to reach ready-state
		& \begin{itemize}
			\item We’re going to wait until the pan is hot enough.
			\item which means that when it has a crackling effect, it is hot enough to start frying stuff.
			\item I  will take this (spoon), I will put some butter in it ... and I leave it on the stove like this on the hot stove ... 
		\end{itemize} \\ \cline{2-3}
		
		& Uses multiple tools simultaneously to achieve goals
		& \begin{itemize}
			\item Take the lid to close it from the top.
			\item I will take this spatula to scoop everything up until everything is inside.
			\item I'm taking now my spatula to help me stir and cook the chicken.
		\end{itemize} \\ \hline
		
	\end{tabular}
	
\end{table}

\begin{table}[H]
	\label{tab:themetable2}
	\centering
	\renewcommand{\arraystretch}{1.4}
	\begin{tabular}{|p{0.25\linewidth}|p{0.25\linewidth}|p{0.4\linewidth}|}
		\hline
		\textbf{Theme} & \textbf{Pattern Name} & \textbf{Example Encodings} \\ \hline
		
		\multirow{3}{=}{Coordinating cooking workflows under constrains} 
		
		& Unintended delays due to space or safety
		& \begin{itemize}
			\item [video] turns timer knob on dryer, realises switch is not on, turns on switch and then timer again.
			\item [video] pushes plates and bowls into one side while discarding contents inside used bowl.
			\item [video] delay between positioning hand and knife before chopping vegetables.
		\end{itemize} \\ \cline{2-3}
		
		& Prioritises decision-making to ensure better workflow
		& \begin{itemize}
			\item [video] holding spatula for a moment before placing it down into the wok, facial expression is “deciding”
			\item [video] tossing vegetables up and down with care when stir-frying, facial expression is focused.
			\item [video] readies the oil to be poured in, but stops suddenly.
		\end{itemize} \\ \cline{2-3}
		
		& Pro-actively organising during cooking to maintain workflow
		& \begin{itemize}
			\item maybe if I feel like it, I will take some yogurt.
			\item While that cooks, Im going to put the rest of the chicken on the fridge. I’m going to close it with a lid and put it in the fridge.
			\item There are two sides on this electronic stove, so I will usually use the right side to fry items
		\end{itemize} \\ \hline
		
		\multirow{1}{=}{Delaying appliance use to be accomplished later}
		& Appliances are delayed intentionally for later use in favour for another action
		& \begin{itemize}
			\item The spatula I will put it into the inside the wok so that I will maybe i'll still need afterwards
			\item Also, I'm going to leave my pan and cutting board i used to cut chicken behind since I have a dishwasher.
		\end{itemize} \\ \hline  
		
	\end{tabular}
	
\end{table}

\begin{table}[H]
	\label{tab:themetable3}
	\centering
	\renewcommand{\arraystretch}{1.4}
	\begin{tabular}{|p{0.25\linewidth}|p{0.25\linewidth}|p{0.4\linewidth}|}
		\hline
		\textbf{Theme} & \textbf{Pattern Name} & \textbf{Example Encodings} \\ \hline
		
		\multirow{3}{=}{Applying knowledge, experience and strategies to maintain control}
		& Relying on experience-based judgement for quick measurements
		& \begin{itemize}
			\item Im just going to put a generous amount of oil.
			\item I will pour the right amount of oil
			\item This is quite difficult to say because normally when you are used to cooking like almost every single day, you would basically know what's the right amount of oil
			\item You keep this, it keeps stirring, and after some time, then I will put some more spices, ... after a certain period of time, you feel like this is pretty much done
		\end{itemize} \\ \cline{2-3}
		
		& Relying on precise measurements with appliances
		& \begin{itemize}
			\item This is about 1 tablespoon (Premium dark soy sauce), that much.
			\item Im going to use a third of it for my lunch today, about 200 grams of chicken.
			\item You will add like probably another one and 1/4 of the salt into your veggies so that it has a taste
			\item The right amount of salt I will put is like half, half the small teaspoon here
		\end{itemize} \\ \cline{2-3}
		
		& Tendency to prepare components beforehand to prevent errors and ensure smooth cooking
		& \begin{itemize}
			\item I will go to my room and bring the cooking utensils.
			\item I  take out some butter from my cupboard, take out some butter, ...
			\item Will put them into a small bowl like this
			\item Open the cabinet below the sink and then I will take out big metal huge bowl
		\end{itemize} \\ \hline
		
	\end{tabular}
	
\end{table}

\begin{table}[H]
	\label{tab:themetable4}
	\centering
	\renewcommand{\arraystretch}{1.4}
	\begin{tabular}{|p{0.25\linewidth}|p{0.25\linewidth}|p{0.4\linewidth}|}
		\hline
		\textbf{Theme} & \textbf{Pattern Name} & \textbf{Example Encodings} \\ \hline
		
		\multirow{2}{=}{Monitoring progress through senses and evaluation}
		
		& Uses sensory cues to determine food quality
		& \begin{itemize}
			\item You can see its that kind of burning because the heat was too high. But it’s fine, we’re going to lower the heat again. We’re just going to occasionally stir while we wait.
			\item I take some water and with a spoon, I try to do this to see if the pressure is leaking because you will see bubbles if the pressure is leaking.
			\item The heat may be too high, so Im going to put it on a medium-high heat. Im just going to keep stirring so it doesnt burn too much. .
			\item [video] puts spoon close to the nose, smells it, and then puts into the mouth to taste.
		\end{itemize} \\ \cline{2-3}
		
		& Critically assessing cooking outcomes for desireed quality
		& \begin{itemize}
			\item Now I’m going to cube or dice it very thinly and evenly.
			\item Because the breast of the chicken is too big, im going to cut it into some parts again. To make it more evenly with other parts of the chicken. So I’m going to cut it into small pieces.
			\item Put the salt in ... then it's time for the taste test.
			\item 900 is usually the middle point, the medium heat where you won't, oh you won't burn the garlic.
		\end{itemize} \\ \hline
		
	\end{tabular}
	
\end{table}

\begin{table}[H]
	\label{tab:themetable5}
	\centering
	\renewcommand{\arraystretch}{1.4}
	\begin{tabular}{|p{0.25\linewidth}|p{0.25\linewidth}|p{0.4\linewidth}|}
		\hline
		\textbf{Theme} & \textbf{Pattern Name} & \textbf{Example Encodings} \\hline
		
		\multirow{5}{=}{Managing risks and hygiene during interaction}
		& Hygiene concerns on appliances prioritises frequent sink use
		& \begin{itemize}
			\item We soak about like twice to make sure that there is actually no more oily residue
			\item I will wash the chopsticks, spoons, all the small and easier cutlery
			\item [video] rubs the wok to check if there is still an oily surface
		\end{itemize} \\ \cline{2-3}
		
		& Hygiene concerns on self pushes for frequent sink use
		& \begin{itemize}
			\item Wash my hands to make sure that it's clean and nice
			\item wipe my hands with the handcloth
			\item will wash my hands and then wash the knife as well to make here clear
		\end{itemize} \\ \cline{2-3}
		
		& Hygiene concerns on food pushes for frequent sink use 
		& \begin{itemize}
			\item wash out all the excessive maybe like pesticides
			\item wash away all the liquids, the soap and then it is time to chop the veggie
			\item put some rice, and with four or five times I rinse it in the water.
		\end{itemize} \\ \cline{2-3}
		
		& Preference towards cooking appliances due to risks
		& \begin{itemize}
			\item the stove in my house is electronic so it it won't be such a hassle
			\item the stove ... I fry anything because it is more safer than the gas option
		\end{itemize} \\ \cline{2-3}
		
		& Compensatory actions while cooking
		& \begin{itemize}
			\item then I will take this long, take this metal knife which is like very sharp so have to be careful
			\item ..The wok can be quite challenging because it is quite heavy to wash
			\item my pressure cooker is now kind of old, which is why I need to check whether it's working properly or not.
		\end{itemize} \\ \hline 
		
	\end{tabular}
	
\end{table}



%%
%% The acknowledgments section is defined using the "acks" environment
%% (and NOT an unnumbered section). This ensures the proper
%% identification of the section in the article metadata, and the
%% consistent spelling of the heading.
\begin{acks}
	We thank the participants for providing us their time to participate in this research. Special recognition goes to Professor Eva and Margarita Osipova for their guidance on this research study.
\end{acks}

%%
%% The next two lines define the bibliography style to be used, and
%% the bibliography file.
\bibliographystyle{ACM-Reference-Format}
\bibliography{main-base}

%%
%% If your work has an appendix, this is the place to put it.
% \appendix


\end{document}
\endinput
%%
%% End of file `sigconf-main.tex'.